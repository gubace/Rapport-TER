\makeatletter
%--------------------------------------------------------------------------------
\usepackage[T1]{fontenc} % font type
\usepackage[french]{babel} % language
\usepackage{lmodern} % font type
\usepackage[shortlabels]{enumitem}
%\setlist[itemize,1]{label={\color{gray}\small \textbullet}} % customises itemize default -
\usepackage{fancyhdr} % customises head and foot-notes
\usepackage{centernot} % allows centering \not with \centernot
\usepackage{stmaryrd} % allows \llbracket
\usepackage[overload]{abraces} % allows \aoverbrace
\usepackage{float}
\usepackage{tcolorbox}
\usepackage{graphicx} % Required for inserting images
\usepackage{subcaption} % Pour gérer les sous-figures



\usepackage[backref=true]{biblatex}

%\usepackage{xcolor} % colour customisation, extends to tables with {colortbl}
\definecolor{astral}{RGB}{46,116,181}
\definecolor{verdant}{RGB}{96,172,128}
\definecolor{algebraic-amber}{RGB}{255,179,102} % definition colour
\definecolor{calculus-coral}{RGB}{255,191,191} % exercice colour
\definecolor{divergent-denim}{RGB}{130,172,211} % proposition colour 
\definecolor{matrix-mist}{RGB}{204,204,204} % remark colour
\definecolor{numeric-navy}{RGB}{204,204,204} % theorem colour 
\definecolor{quadratic-quartz}{RGB}{204,153,153} % example colour 

\usepackage[utf8]{inputenc}  % LaTeX, comprends les accents !
\usepackage[T1]{fontenc}  % Police contenant les caractères français
\usepackage[french]{babel}  % Le document est en français
\usepackage{fullpage}  % pour les marges
\usepackage{lmodern}  % font type
\usepackage{datetime}  % pour la date
\newdate{frontpagedate}{12}{05}{2024}  % jour, mois, année
\usepackage[shortlabels]{enumitem}  % pour les listes numérotées
\setlist[itemize,1]{label={\color{gray}\small \textbullet}} % customises itemize default -
%\usepackage{xcolor} % colour customisation, extends to tables with {colortbl}
\usepackage{url}  % pour les url
%\usepackage[breaklinks]{hyperref} % pour casser les liens hypertextes
%\usepackage{breakurl} % pour casser les liens hypertextes
\usepackage{amsmath} % pour les formules mathématiques



\usepackage{float}
\usepackage{caption}
\usepackage{latexsym}
\usepackage{amsmath}
\usepackage{amsfonts}
\usepackage{amssymb}
\usepackage{dsfont}
\usepackage{amsthm}
\usepackage{mathtools}
\usepackage{mathrsfs}
\usepackage{MnSymbol}
\usepackage{etoolbox}% http://ctan.org/pkg/etoolbox

\usepackage[table,xcdraw,svgnames]{xcolor}
\usepackage[colorlinks=true, linkcolor=blue, urlcolor=blue, citecolor=blue, pdfborder={0 0 0}]{hyperref}

\usepackage[backend=biber,style=numeric,backref=true]{biblatex}
\addbibresource{biblio.bib}

\usepackage{tikz}
\usepackage{pgfplots}
\pgfplotsset{compat=1.18}
\usetikzlibrary{arrows}
\usetikzlibrary{trees, positioning}


\newtheoremstyle{gen-style}{\topsep}{\topsep}%
{}%         Body font
{}%         Indent amount (empty = no indent, \parindent = para indent)
{\sffamily\bfseries}% Thm head font
{.}%        Punctuation after thm head
{ }%     Space after thm head (\newline = linebreak)
{\thmname{#1}\thmnumber{~#2}\thmnote{~#3}}%         Thm head spec


\newtheoremstyle{no-num-style}{\topsep}{\topsep}%
{}% Body font
{}% Indent
{\sffamily\bfseries}% Head font
{.}% Punctuation
{ }% Space after head
{\thmname{#1}\thmnote{~#3}}


\usepackage[]{mdframed}

\newcommand{\mytheorem}[5]{%
	\ifstrequal{#5}{o}{%
		\newmdtheoremenv[
		hidealllines=true,
		leftline=true,
		skipabove=0pt,
		innertopmargin=-5pt,
		innerbottommargin=2pt,
		linewidth=4pt,
		innerrightmargin=0pt,
		linecolor=#3,
		]{#1}[#4]{#2}%
	}{%
		\newmdtheoremenv[
		hidealllines=true,
		leftline=true,
		skipabove=0pt,
		innertopmargin=-5pt,
		innerbottommargin=2pt,
		linewidth=4pt,
		innerrightmargin=0pt,
		linecolor=#3,
		]{#1}{#2}[#4]%
	}%
}

\theoremstyle{gen-style}
\mytheorem{proposition}{Proposition}{divergent-denim}{section}{}
\mytheorem{propdef}{Proposition - Définition}{divergent-denim}{section}{}
\mytheorem{theorem}{Théorème}{quadratic-quartz}{section}{}
\mytheorem{lemme}{Lemme}{quadratic-quartz}{section}{}
\mytheorem{corollaire}{Corollaire}{quadratic-quartz}{section}{}
\mytheorem{example}{Exemple}{quadratic-quartz}{section}{}
\mytheorem{remark}{Remarque}{matrix-mist}{section}{}
\mytheorem{notation}{Notation}{matrix-mist}{section}{}
\mytheorem{exercise}{Exercice}{calculus-coral}{section}{}
\mytheorem{exercice}{Exercice}{calculus-coral}{section}{}
\mytheorem{definition}{Definition}{algebraic-amber}{section}{}

\newcounter{oc-counter}
\mytheorem{oc-proposition}{Proposition}{divergent-denim}{oc-counter}{o}
\mytheorem{oc-propdef}{Proposition - Définition}{divergent-denim}{oc-counter}{o}
\mytheorem{oc-theorem}{Théorème}{divergent-denim}{oc-counter}{o}
\mytheorem{oc-lemme}{Lemme}{quadratic-quartz}{oc-counter}{o}
\mytheorem{oc-example}{Exemple}{quadratic-quartz}{oc-counter}{o}
\mytheorem{oc-remark}{Remarque}{matrix-mist}{oc-counter}{o}
\mytheorem{oc-exercise}{Exercice}{calculus-coral}{oc-counter}{o}
\mytheorem{oc-definition}{Definition}{algebraic-amber}{oc-counter}{o}

\theoremstyle{no-num-style}
\mytheorem{td-sol}{Solution}{verdant}{}{}
\mytheorem{no-num-example}{Exemple}{quadratic-quartz}{}{}
\mytheorem{no-num-definition}{Définition}{algebraic-amber}{}{}
\mytheorem{no-num-proposition}{Proposition}{divergent-denim}{}{}
\mytheorem{no-num-theorem}{Théorème}{algebraic-amber}{}{}
\mytheorem{no-num-result}{Résultat}{algebraic-amber}{}{}
\mytheorem{no-num-lemma}{Définition}{quadratic-quartz}{}{}
\mytheorem{no-num-remark}{Remarque}{matrix-mist}{}{}
\mytheorem{no-num-exercice}{Exercice}{calculus-coral}{}{}
\mytheorem{no-num-explication}{Explication}{calculus-coral}{}{}
\mytheorem{oc-intro}{Introduction}{quadratic-quartz}{}{}
\mytheorem{oc-proof}{Preuve}{verdant}{}{}
\mytheorem{oc-young}{Formule de Taylor à l'ordre 2}{verdant}{}{}
\mytheorem{oc-notation}{Notation}{matrix-mist}{}{}
\mytheorem{rappel}{Rappel}{matrix-mist}{section}{}
\mytheorem{myproof}{Preuve}{verdant}{}{}
\mytheorem{td-exo}{Exercice}{calculus-coral}{}{}
\numberwithin{oc-counter}{subsection}

%---------------
% Mise en page
%--------------

\setlength{\parindent}{0pt}

\providecommand{\defemph}[1]{{\sffamily\bfseries\color{astral}#1}}


\usepackage{sectsty}
\allsectionsfont{\color{astral}\normalfont\sffamily\bfseries}

\usepackage{mathrsfs}

%----- Easy way to redeclare math operators -----
\makeatletter
\newcommand\RedeclareMathOperator{%
	\@ifstar{\def\rmo@s{m}\rmo@redeclare}{\def\rmo@s{o}\rmo@redeclare}%
}
\newcommand\rmo@redeclare[2]{%
	\begingroup \escapechar\m@ne\xdef\@gtempa{{\string#1}}\endgroup
	\expandafter\@ifundefined\@gtempa
	{\@latex@error{\noexpand#1undefined}\@ehc}%
	\relax
	\expandafter\rmo@declmathop\rmo@s{#1}{#2}}
\newcommand\rmo@declmathop[3]{%
	\DeclareRobustCommand{#2}{\qopname\newmcodes@#1{#3}}%
}
\@onlypreamble\RedeclareMathOperator
\makeatother

\newcommand{\skipline}{\vspace{\baselineskip}}
\newcommand{\noi}{\noindent}
%------------------------------------------------


\newcommand{\adh}[1]{\mathring{#1}} %adherence
\newcommand{\badh}[1]{\mathring{\overbrace{#1}}} % big adherence
\newcommand{\norme}{\mathcal{N}} % norme
\newcommand{\ol}[1]{\overline{#1}} % overline
\newcommand{\ul}[1]{\underline{#1}} % underline
\newcommand{\sub}{\subset} % subset
\newcommand{\scr}[1]{\mathscr{#1}} % scr rapide
\newcommand{\bb}[1]{\mathbb{#1}} % bb rapide
\newcommand{\bolo}[1]{B({#1}\mathopen{}[\mathclose{}} % boule ouverte
\newcommand{\bolf}[1]{B({#1}\mathopen{}]\mathclose{}} % boule fermee
\newcommand{\act}{\circlearrowleft} % agit sur
\newcommand{\glx}[1]{\text{GL}_{#1}} % GL_x
\newcommand{\cequiv}[1]{\mathopen{}[#1\mathclose{}]} % classe d'equivalence
\newcommand{\restr}[2]{#1\mathop{}\!|_{#2}} % restriction
\newcommand{\nn}[1]{||#1||}
\newcommand{\n}[1]{|#1|}


%----- Intervalles -----
\newcommand{\oo}[1]{\mathopen{]}#1\mathclose{[}}
\newcommand{\of}[1]{\mathopen{]}#1\mathclose{]}}
\newcommand{\fo}[1]{\mathopen{[}#1\mathclose{[}}
\newcommand{\ff}[1]{\mathopen{[}#1\mathclose{]}}
\newcommand{\boo}[1]{\mathopen{\big]}#1\mathclose{\big[}}
\newcommand{\bof}[1]{\mathopen{\big]}#1\mathclose{\big]}}
\newcommand{\bfo}[1]{\mathopen{\big[}#1\mathclose{\big[}}
\newcommand{\bff}[1]{\mathopen{\big[}#1\mathclose{\big]}}
\newcommand{\norm}[1]{\mathopen{}\left\|#1\right\|\mathclose{}} % norme


\providecommand{\1}{\mathds{1}}
\DeclareMathOperator{\im}{\mathsf{Im}}
\DeclareRobustCommand{\re}{\mathsf{Re}}
\RedeclareMathOperator{\ker}{\mathsf{Ker}}
\RedeclareMathOperator{\det}{\mathsf{det}}
\DeclareMathOperator{\vect}{\mathsf{Vect}}
\DeclareMathOperator{\diam}{\mathsf{Diam}}
\DeclareMathOperator{\orb}{\mathsf{orb}}
\DeclareMathOperator{\st}{\mathsf{st}}
\DeclareMathOperator{\spr}{\mathsf{SP_{\bb R}}}
\DeclareMathOperator{\aut}{\mathsf{Aut}}
\DeclareMathOperator{\bij}{\mathsf{Bij}}
\DeclareMathOperator{\rank}{\mathsf{rank}}
\DeclareMathOperator{\tr}{\mathsf{tr}}
\DeclareMathOperator{\id}{\mathsf{Id}}
\providecommand{\B}{\mathsf{B}}


\providecommand{\dpar}[2]{\frac{\partial #1}{\partial #2}}
\makeatother
